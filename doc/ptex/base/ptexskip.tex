\documentclass{jarticle}
\ProvidesFile{ptexskip.tex}[1997/08/08 v0.4 about pTeX skips]
\usepackage{plext}

\title{p\TeX~2.1.5における数式の前後}
\author{中野 賢\footnote{Ken Nakano (\texttt{<ken-na at ascii.co.jp>}):
  株式会社 アスキー 出版技術部}}
\date{1997年8月8日}

\newcommand{\hex}[1]{\hskip\xkanjiskip\texttt{"#1}}
\newcommand{\cs}[1]{\hskip\xkanjiskip\texttt{\char"5C #1}}
%\xspcode"5C=1
\xspcode"22=2 % 2
\xspcode"27=2 % 2
\xspcode"5C=1 % 1
\xspcode"60=1 % 1
\prebreakpenalty"22=10001  % ''
\prebreakpenalty"27=10002  % '
\postbreakpenalty"5C=10003 % ``, \
\postbreakpenalty"60=10004  % `

% サンプル用
\newdimen\dimenA

%\tracingcommands=2
%\tracingmacros=2
%\tracingparagraphs=2
%\tracingoutput=2
\showboxbreadth=\maxdimen
\showboxdepth=\maxdimen
\nonstopmode

\begin{document}
\maketitle

\section*{はじめに}
p\TeX{}では、和文と欧文の間に\emph{四分アキ}と呼ばれるスペースを
自動的に挿入するようになっている。挿入される箇所は、和文と欧文の間という
だけでなく、他にも次の条件が揃っていなくてはならない。

\begin{itemize}
\item \cs{autoxspcode}の状態
\item \cs{xspcode}および\cs{inhibitxspcode}による抑制を受けない
\item shift\_amount(移動量)がゼロである\cs{hbox}の直前、直後
\item 合字、ペナルティ、ベースライン調整量、暗黙の\cs{kern}の直前、直後
\end{itemize}

本文では多くの場合、単純に文字が並んでいるだけなので、このような条件を
知らなくても、期待している結果を得ることができる。
しかし、数式を組み立てた場合、自分ではとくに指定をしていなくても、
\TeX{}が内部処理で、文字の位置を調整したり、\cs{vbox}に入れたりするために、
四分アキが入らず、その原因もわからないということがある。

そこで、p\TeX~2.1.5では、テキスト数式の前後への四分アキの挿入方法を変更した。
この文書では、その拡張に際しての仕様の変更について簡単に説明をしている。

\section{p\TeX~2.1.4までの数式の前後}
p\TeX~2.1.4で\footnote{より正確に言えば、p\TeX~2.1.5 $\beta$~6まで}、
テキスト数式の前や後に四分アキが入らない主な原因は以下の3つである。

\begin{itemize}
\item \cs{xspcode}による影響の場合
\item シフトされた\cs{hbox}となる場合
\item \cs{vbox}となる場合
\end{itemize}

\subsection{\cs{xspcode}による影響の場合}
\cs{xspcode}および\cs{inhibitxspcode}による抑制は、たとえば、
$\alpha$や$\beta$の前後にスペースが入らないという結果をもたらす。
なぜならば、$\alpha$はcmmi10の\hex{0B}の位置、$\beta$は\hex{0C}の位置に
あるからである。これら文字の\cs{xspcode}の値はゼロ、すなわち前後への
四分アキを抑制するという動作をする。
その結果、前後に四分アキが入らないのである。

\subsection{シフトされた\cs{hbox}となる場合}
shift\_amountがゼロでない\cs{hbox}の例としては、
$\sum$や$\int$や$x^2$や$x_2$や$\sqrt{\hbox{Var(X)}}$などを挙げられる。

$\sum$はcmex10の\hex{50}の位置にあり、これはcmr10の``P''と同じ位置
なので\cs{xspcode}による抑制は受けない。しかし、位置の調整のために、
\cs{hbox}に入れられ、左にシフトされる。この結果、\hex{50}の文字が
一つだけにも関わらず、処理対象外の\cs{hbox}となってしまい、
前後への四分アキが入らなくなる。$\int$も同様である。

$x^2$や$x_2$などの場合は、上付き/下付き文字がシフトされた\cs{hbox}として
組み立てられているので、``$x$''の前には四分アキが入るが、
``2''の後ろには入らないことになる。

根号記号の場合は、全体がシフトされた\cs{hbox}として組み立てられるので、
前後に四分アキが入らない。

\subsection{\cs{vbox}となる場合}
その他に四分アキが入る箇所は、
合字、ペナルティ、ベースライン調整量、暗黙の\cs{kern}の直前か直後
だけである。これ以外の箇所には入らない。
数式の前後に限れば、このうちの、\cs{vbox}の前後には入らないという
制限が大きく影響をしている。たとえば、$\frac{1}{k}$といった分数や、
$x_k^2$のように上付きと下付き文字の両方を指定したときが、これに当たる。

分数の場合は、最終的に\cs{vbox}として組み立てられる。
そのため、分数の前後に四分アキが入らなくなる。
上付きと下付きの両方がある場合は、それらが\cs{vbox}に入れられるため、
入らなくなるのである。上付きと下付きの一方の場合と状況は異なるが、
結果は一緒である。

\section{p\TeX~2.1.5 $\beta7$での数式の前後}
今までの実装コードでは、数式の開始や終了のノードを見つけても、
開始/終了ノードを単純にスキップするだけで、その後、数式の内部の文字を見て、
スペースを挿入するかどうかを判断をしていた。そのために数式を囲んでいる
\cs{hbox}がシフトしていたり、\cs{vbox}であったりすると前後にスペースが
入らないという結果になっていた。

そこで、$\beta7$では、数式の開始ノードを見つけたとき、次の要素が何で
あるかを調べることにした。

数式の次の要素が``文字''そのものである場合、テキストの最後の文字との関係で
適切なスペースを入れる。ここで、従来の動作と異なるのは、数式内部の文字の
\cs{xspcode}あるいは\cs{inhibitxspcode}の設定を無視するようにした点である。
つまり、数式内の文字とその直前との文字が和文か欧文かだけを調べて、
四分アキか漢字間スペースを挿入するようにした。
この変更で$\alpha$や$\beta$などの文字の前にスペースが入るようになる。

数式の開始ノードの次の要素が文字以外の場合は、それが何であっても
四分アキを入れる。この修正によって、数式がシフトされた\cs{hbox}や
\cs{vbox}の場合であっても四分アキが入る。
このようなボックスになるときは、上付き/下付き文字、分数、数学記号を
使った場合などである。

終了ノードの時点では、この後に挿入するスペースが決まっていればそれを用いる。
すなわち、数式の最後とその後るの文字がともに和文であれば漢字間スペースを
挿入し、いずれかが欧文であれば四分アキを挿入する。
このときも、数式内部の最後の文字の\cs{xspcode}も無視するようにし、
$\alpha$や$\beta$などの後ろにも入るようにしている。

シフトされた\cs{hbox}や\cs{vbox}などで終了したときは、
前側に四分アキを入れたので、合わせる意味で後側にも四分アキを挿入する。
これで$x^2$や$\frac{dx}{dy}$や$\sqrt{3}$などの後ろにもスペースが入る。

なお、数式の直前の文字と数式内部の先頭文字がともに欧文の場合、
あるいは数式内部の最後と数式直後がともに欧文の場合は、
\TeX{}と同じであり特別なスペースを入れないのは従来と同じである。

\section{p\TeX~2.1.5 $\beta8$での数式の前後}
p\TeX~2.1.5 $\beta8$では、単純にテキスト数式の前後に四分アキを入れる方法
で実装している。ただし、「(」や「)」などによる抑制を効かせるべき箇所には
四分アキは入らない。つまり
``\verb|($y=a+b$)|''は
``($y=a+b$)''となり
``(\kern\xkanjiskip$y=a+b$\kern\xkanjiskip)''とは
ならないことに注意する。
`\verb|($y=a+b$)|'は
`($y=a+b$)'となり
`(\kern\xkanjiskip$y=a+b$\kern\xkanjiskip)'とは
ならないことに注意する。
\showlists

$\beta7$版と$\beta8$版との仕様の違いは、
次のように数式内の文字が漢字で始まる(あるいは終わる)ときに現れる。
\begin{verbatim}
    \kanjiskip=4pt \xkanjiskip=8pt
    □□$表面積=4\pi r^2$□□
\end{verbatim}
この例をそれぞれで処理した場合、
\begin{quote}
\kanjiskip=4pt \xkanjiskip=8pt
□□\kern\kanjiskip$表面積=4\pi r^2$\kern\xkanjiskip □□
  \quad($\beta7$\kern.25zw 版)\\
□□\kern\xkanjiskip$表面積=4\pi r^2$\kern\xkanjiskip □□
  \quad($\beta8$\kern.25zw 版)\\
□□$表面積=4\pi r^2$□□
  \quad\hbox{\kanjiskip=0pt\xkanjiskip=.25zw (この版)}
\end{quote}
となる。つまり、$\beta7$版では数式内部の文字と直前の文字を比較しているため、
前側に漢字間スペースが入り、後側に四分アキが入る。
一方、$\beta8$版では数式内部の文字に関係なく、両側に四分アキが入る。

なお、両者の違いは、次のように\cs{hbox}を用いたときにも現れる。
\begin{verbatim}
    \kanjiskip=4pt \xkanjiskip=8pt
    □□$\hbox{表面積}$□□
\end{verbatim}
これを処理すると、$\beta7$版は目指すべき仕様とは異なり、
後ろに何も入らない(バグ)。$\beta8$版では仕様どおりに両側に四分アキが入る。
\begin{quote}
\kanjiskip=4pt \xkanjiskip=8pt
□□\kern\xkanjiskip$\hbox{表面積}$\kern0pt □□
  \quad ($\beta7$\kern.25zw 版)\\
□□\kern\kanjiskip$\hbox{表面積}$\kern\kanjiskip □□
  \quad\hbox{\kanjiskip=0pt($\beta7$\kern.25zw 版を修正した場合)}\\
□□\kern\xkanjiskip$\hbox{表面積}$\kern\xkanjiskip □□
  \quad ($\beta8$\kern.25zw 版)\\
□□$\hbox{表面積}$□□
  \quad\hbox{\kanjiskip=0pt\xkanjiskip=.25zw (この版)}
\end{quote}

実際には、p\TeX, p\LaTeX のデフォルトの設定では\cs{kanjiskip}と\cs{xkanjiskip}%
の値はもっと小さいので、違いはそれほど目立たないかもしれない。
しかし、縦組にした場合は、つぎのような結果となる。
\begin{quote}
\begin{minipage}<t>{16zw}
△△\kern\kanjiskip$表面積=4\pi r^2$\kern\xkanjiskip\raisebox{-.25zw}{△△}
  \quad($\beta7$\kern.25zw 版)\\
\\
△△\kern\kanjiskip$表面積=4\pi r^2$\kern\xkanjiskip △△\\
  \hfill($\beta7$\kern.25zw 版を修正した場合)\\
\\
△△\kern\xkanjiskip$表面積=4\pi r^2$\kern\xkanjiskip △△
  \quad($\beta8$\kern.25zw 版)\\
\\
△△$表面積=4\pi r^2$△△
  \quad\hbox{\kanjiskip=0pt\xkanjiskip=.25zw (この版)}
\end{minipage}
\end{quote}
$\beta7$の結果を見ると文字間のスペースは仕様どおりになっている。
しかし、ベースラインの調整量が数式の後ろで戻っていないというバグがある。
たとえバグを修正したとしても、決して読みやすいとは思えない。

以上の結果を比較すると、$\beta8$の挿入方法のほうが自然で、読みやすいように
思える。また、挿入されるスペースも前後の文字、数式内の文字に関係なく、
四分アキと決まっているので結果がわかりやすい。
そこで、$\beta8$では、この節の冒頭で述べた、
\begin{center}
単純に、テキスト数式の前後に四分アキを入れる
\end{center}
という仕様で実装している。

\subsection{別の問題}
$\beta7$の実装では、完全に数式にだけ影響するようにしていなかったため、
通常の文章内でのボックスに対する動作も変わってしまっていた。たとえば、
\begin{quote}
\begin{verbatim}
□□\AA △△△△\hbox to0pt{ABC\hss}○○○○\hbox{}A\hbox{}□□□□
\end{verbatim}
\end{quote}
が
\begin{quote}
\setbox0=\hbox{h}\dimenA=\ht0 \advance\dimenA-1ex
□□
\leavevmode\rlap{\raise.67\dimenA\hbox{\char'27}}\kern\xkanjiskip A%
\kern\xkanjiskip △△△△
\setbox0\hbox{ABC}\dimenA=\wd0\advance\dimenA-\xkanjiskip
\kern\xkanjiskip ABC\kern-\dimenA ○○○○
\kern\xkanjiskip A\kern\xkanjiskip □□□□
\end{quote}
となる。先ほどのリストを処理して期待する結果は、次のようだと思われる。
\begin{quote}
\setbox0=\hbox{h}\dimenA=\ht0 \advance\dimenA-1ex
□□
\kern\xkanjiskip
\leavevmode\rlap{\raise.67\dimenA\hbox{\char'27}}\kern0pt A%
\kern\xkanjiskip △△△△
\setbox0\hbox{ABC}\dimenA=\wd0%
\kern\xkanjiskip ABC\kern-\dimenA ○○○○
\kern0pt A\kern0pt □□□□
\end{quote}
この版では、
\begin{quote}
□□\AA △△△△\hbox to0pt{ABC\hss}○○○○\hbox{}A\hbox{}□□□□
\end{quote}
となる。
\AA の前に四分アキが入っていなければ、それはシフトされたボックスの前後に
アキが入らないからである。$\beta7$の実装で``{\char'27}''と``A''の位置が
ずれているのは、シフトされたボックスの前には入れないのは変わらないけれども、
後ろで必ず入れないようにするのをやめたためである。
また、$\beta7$では、ボックスの中身の最後が文字以外でも、空でも四分アキの
挿入に影響をさせないようにしていたため、幅がゼロのボックスを作成しても、
参照点の位置が戻らないし、処理の抑制もされなくなっていた
\footnote{これらの問題は$\beta11$で修正。}。

\appendix
\section{自動挿入されるスペースの種類}
p\TeX{}では、日本語文書をきれいに組版するために、
和文と和文の間、和文と欧文の間に自動的にスペースを入れるように拡張している。
欧文と欧文の間は、\TeX{}のメカニズムそのままである。

和文と和文の間に入れるスペースの量は、\cs{kanjiskip}という長さレジスタに
設定をする。和文と欧文との間は\cs{xkanjiskip}という長さレジスタである。
p\TeX{}のデフォルトでは、
\begin{verbatim}
    \kanjiskip=0pt plus .4pt minus .4pt
    \xkanjiskip=.25zw plus 1pt minus 1pt
\end{verbatim}
となっている。この設定は、
\cs{kanjiskip}は標準でゼロポイント、場合によって$\pm0.4$ポイント分だけ伸縮、
\cs{xkanjiskip}は標準でその時点の和文フォントの幅の$1/4$、
場合によって$\pm1$ポイント分だけ伸縮しても良いということを意味している。

\cs{kanjiskip}も\cs{xkanjiskip}も、段落の終わりか、\cs{hbox}の最後の時点
の値が有効となる。したがって、ひとつの段落内や\cs{hbox}内で複数回指定を
しても、その最後の指定によって処理される。

ただし「、」や「(」のように、特定の文字が連続する場合、そのまま全角幅
で並べ、間に\cs{kanjiskip}を挿入すると文字の間が離れすぎてしまう。
このときには、\cs{kanjiskip}ではなく、JFM(Japanese Font Metric)で設定
されているスペースの量が使われる。

\cs{xkanjiskip}に関しても、和文と「;」、「(」と和文、欧文と「。」、
「…」と前後の欧文のような箇所には、スペースを挿入しないほうがきれいに
見える。そこで、特定の欧文の文字に対して、\cs{xkanjiskip}の挿入を制御
するために\cs{xspcode}が用意されている。
特定の和文に対しては\cs{inhibitxspcode}を用いて制御する。


\section{スペースに関するプリミティブ}
p\TeX{}で拡張した、スペースに関するプリミティブは以下のとおり。

\subsection{\cs{kanjiskip}, \cs{autospacing}, \cs{noautospacing}}
\cs{kanjiskip}は、
漢字と漢字の間に自動的に挿入するスペースの量を格納する長さレジスタである。

\cs{autospacing}と\cs{noautospacing}は、
漢字と漢字の間にスペースを挿入するかどうかを指定するのに用いる。
\cs{autospacing}を指定すると自動的に挿入される。
\cs{noautospacing}を指定すると漢字間へのスペース挿入は抑制される。

\subsection{\cs{xkanjiskip}, \cs{autoxspacing}, \cs{noautoxspacing}}
\cs{xkanjiskip}は、
漢字と英字の間に自動的に挿入するスペースの量を格納する長さレジスタである。

\cs{autoxspacing}と\cs{noautoxspacing}は、
漢字と英字の間にスペースを挿入するかどうかを指定するのに用いる。
\cs{autoxspacing}を指定すると自動的に挿入される。
\cs{noautoxspacing}を指定すると漢字間へのスペース挿入は抑制される。

\subsection{\cs{xspcode}}
\cs{xspcode}は、指定した英字と漢字との間のスペース挿入を
どのように\textgt{許可}するかの設定をするプリミティブである。
動作は、つぎのいずれかの数値で指定をする。

\begin{center}
\begin{tabular}{ll}
0 & 前後の漢字との間へのスペースの挿入を禁止する。\\
1 & 直前の漢字との間にだけスペースの挿入を許可する。\\
2 & 直後の漢字との間にだけスペースの挿入を許可する。\\
3 & 前後の漢字との間にスペースの挿入を許可する。\\
\end{tabular}
\end{center}

初期値は、\texttt{[0-9A-Za-z]}は3、それ以外はゼロになっている。
ただし、以下の文字については、kinsoku.texで別の値に初期化されている。
括弧内はASCII文字コード(16進数)である。

\begin{center}
\begin{tabular}{l@{\hspace{2.5zw}}l@{\hspace{2.5zw}}l@{\hspace{2.5zw}}l}
\texttt{(}=1 (\hex{28}) & \texttt{)}=2 (\hex{29}) &
\texttt{[}=1 (\hex{5B}) & \texttt{]}=2 (\hex{5D})\\
\texttt{`}=1 (\hex{60}) & \texttt{'}=2 (\hex{27}) &
\texttt{;}=2 (\hex{3B}) & \texttt{,}=2 (\hex{2C})\\
\texttt{.}=2 (\hex{2E}) \\
\end{tabular}
\end{center}

ただし、\cs{xspcode}の設定は、文字コードに対してであり、フォントによって
異なる値を指定することはできない。したがって\hex{60}の位置にある文字は、
cmr10では``\textrm{`}'', cmmi10では``$\ell$'', cmex10では``$\coprod$''で
あるが、これらの文字はすべて\cs{xspcode}=1として処理される。

\subsection{\cs{inhibitxspcode}}
\cs{inhibitxspcode}は、指定した漢字と英字の間のスペース挿入を
どのように\textgt{抑制}するかの設定をするプリミティブである。
動作は、つぎのいずれかの数値で指定をする。

\begin{center}
\begin{tabular}{ll}
0 & 漢字と英字との間のスペースの挿入を禁止する。\\
1 & 直前の英字との間のスペースの挿入を禁止する。\\
2 & 直後の英字との間のスペースの挿入を禁止する。\\
3 & 前後の英字との間のスペースの挿入を許可する。\\
\end{tabular}
\end{center}

初期値は、すべての漢字について3である。
ただし、以下の文字については、kinsoku.texで別の値に初期化されている。

\begin{center}
\begin{tabular}{l@{\hspace{2.5zw}}l@{\hspace{2.5zw}}l@{\hspace{2.5zw}}
                l@{\hspace{2.5zw}}l@{\hspace{2.5zw}}l}
、=1 & 。=1 & ,=1 & .=1 & ;=1 & ?=1\\
(=2 & )=1 & [=2 & ]=1 & {=2 & }=1\\
‘=2 & ’=1 & “=2 & ”=1 & 〔=2 & 〕=1\\
<=2 & >=1 & 《=2 & 》=1 & 「=2 & 」=1\\
『=2 & 』=1 & 【=2 & 】=1 & −=0 & 〜=0\\
…=0 & ¥=0 & °=0 & ′=1 & ″=1\\
\end{tabular}
\end{center}

\end{document}
