%% <2001/10/04>
\documentclass{plnews}

\publicationyear{2004}% 発行年
\publicationmonth{08}% 発行月
\publicationissue{8}% 番号
\author{田中 禎之(\texttt{<sada-t at ascii.co.jp>})
}

\begin{document}

\maketitle

\section{この文書について}
この文書は、p\LaTeXe{}\texttt{<2004/08/10>}版について、
前回の版(\texttt{<2001/09/04>})からの更新箇所をまとめたものです。
それ以前の変更点については、\file{plnews*.tex}や\file{Changes.txt}を
参照してください。
\LaTeX{}レベルでの更新箇所は、\LaTeX{}に付属の\file{ltnews*.tex}などを
参照してください。


\section{和文エンコーディング切り替え対応}
これまでは和文エンコーディングにはJY1,JT1のみしか使用できませんでしたが、
新たにエンコーディングを定義して使用できるようにしました。

和文エンコーディングは以下のマクロで定義します。

\begin{itemize}
\item |\DeclareYokoKanjiEncoding{<エンコーディング名>}{<実行コード>}{<実行コード>}|\\
	横書き用の和文エンコーディングを宣言する。
	引数は|\DeclareFontEncoding|と同じ。
\item |\DeclareTateKanjiEncoding{<エンコーディング名>}{<実行コード>}{<実行コード>}|\\
	縦書き用の和文エンコーディングを宣言する。
	引数は|\DeclareFontEncoding|と同じ。
\item |\KanjiEncodingPair{<横エンコーディング名>}{<縦エンコーディング名>}|\\
	横書きと縦書きの和文エンコーディングを関連付ける。
\end{itemize}

横書きと縦書きのエンコーディングは以下のように必ず|\KanjiEncodingPair|で
対応を関連付けてから使用します。
関連付けられたエンコーディングは|\yoko|, |\tate|の実行時に切り替わります。

\begin{verbatim}
  :
\DeclareYokoKanjiEncoding{NY1}{}{}
\DeclareYokoKanjiEncoding{NT1}{}{}
\KanjiEncodingPair{NY1}{NT1}
  :
\begin{document}
  :
\fontencoding{NY1}
  :
\selectfont
  :
\end{verbatim}

現時点では|\DeclareTextCompositeCommand|のようなエンコーディングで動作を
切り替えるためのマクロには対応していません。そのため、フォントの組み合わせを
変更する程度にしか利用できません。

\section{その他の主な修正箇所}
次のような不具合の修正や仕様の変更をしました。

\begin{itemize}
\item |verbatim|環境で余計な前後空きが発生しないように|\fontfamily|を修正。
\item |\ascii|,|\Ascii|,|\ASCII|マクロのエラーを修正。
\item 和文フォントサイズの基準値の設定を修正。
\item 縦組スタイルで|\flushbottom|しないように修正。
\item |\part|,|\chapter|の直後でインデントが発生しないバグを修正。
\item 見出しの前後の空きを調整。
\item \LaTeX 2003/12/01版での動作を確認。
\end{itemize}


\section{フォーマットファイル作成時の注意}
現在のp\TeX{}では、8ビットコードの連続を16ビットコードと認識して
しまう場合があります。そのため、フランス語やキリル文字などの
8ビットコードが連続するハイフンパターンはまず使えせん。
例えばcmcyraltパッケージでは、途中でつぎのようなエラーになります。

\begin{verbatim}
(/usr/local/share/texmf/tex/latex/contrib/
other/cmcyralt/rhyphen.tex Russian hyphena
tion
! Bad \patterns.
l.107  . え
           2
?
\end{verbatim}

このときは、``|?|''のプロンプトに対して``|x|''で終了してください。
残念ながら、このハイフンパターンをp\TeX{}で利用することはできません。

そこで、hyphen.cfgを用意して、不用意に他のハイフンパターンを
読み込まないようにしてあります。詳しくはREADME2.txtをご覧ください。

\section{その他}
p\TeX{}やp\LaTeXe{}に関する最新情報は、p\TeX{}ホームページ
\begin{verbatim}
    http://www.ascii.co.jp/pb/ptex
\end{verbatim}
より、入手することができます。

バグ報告やお問い合わせなどは、電子メールで
\begin{verbatim}
    www-ptex@ascii.co.jp
\end{verbatim}
までお願いします。

\end{document}
