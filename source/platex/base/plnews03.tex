%% <1998/02/17>
\documentclass{plnews}

\publicationyear{1998}
\publicationmonth{2}
\publicationissue{3}
\author{中野 賢(\texttt{<ken-na at ascii.co.jp>})
     \& 富樫 秀昭(\texttt{<hideak-t at ascii.co.jp>})
}

\begin{document}

\maketitle

\section{この文書について}
この文書は、p\LaTeXe{}\texttt{<1998/02/01>}版について、
p\LaTeXe{}\texttt{<1997/07/02>}版からの更新箇所をまとめたものです。

このp\LaTeXe{}は、\LaTeX{}\texttt{<1997/12/01>}版に対応しています。
\LaTeX{}レベルでの更新箇所は、\LaTeX{}に付属のltnewsファイルを
参照してください。

\section{パッチの取り込み}
plpatch.ltxで提供していた、つぎの修正を取り込みました。

\begin{itemize}
\item 小文字のファイル名のフォント定義ファイル(.fdファイル)をロードする
ための修正。
\item |\\|コマンドの直前の空白を無視するための修正。
\end{itemize}

\section{クラスファイルの修正}
日本語クラスファイル(1.1e)に対して、以下の変更を加えました。

\begin{itemize}
\item 縦組クラスで書体の大きさを変更したとき、ベースラインがずれる(1.1f)。
\item onesideオプションを指定したとき、section レベルの文字列が柱に
      出力されない(1.1g)。
\item landscapeオプション指定時のレイアウトパラメータの修正(1.1h)。
\item jreport, jbookクラスで、onesideオプションを指定し、ページスタイルを
      bothstyleにすると、コンパイルエラーになる(1.1i)。
\end{itemize}

\section{フォーマットファイル作成時の注意}
現在のp\TeX{}(p2.1.5)では、8ビットコードの連続を16ビットコードと認識して
しまう場合があります。そのため、フランス語やキリル文字などの
8ビットコードが連続するハイフンパターンはまず使えせん。
例えばcmcyraltパッケージでは、途中でつぎのようなエラーになります。

\begin{verbatim}
(/usr/local/share/texmf/tex/latex/contrib/
other/cmcyralt/rhyphen.tex Russian hyphena
tion
! Bad \patterns.
l.107  . え
           2
?
\end{verbatim}

このときは、``|?|''のプロンプトに対して``|x|''で終了してください。
残念ながら、このハイフンパターンをp\TeX{}で利用することはできません。

p\LaTeXe{}では|$TEXMF/tex/platex/base/|ディレクトリにhyphen.cfgを
用意して、不用意に他のハイフンパターンを読み込まないようにしてあります。


\section{その他}
p\TeX{}やp\LaTeXe{}に関する最新情報は、p\TeX{}ホームページ
\begin{verbatim}
    http://www.ascii.co.jp/pb/ptex
\end{verbatim}
より、入手することができます。

バグ報告やお問い合わせなどは、電子メールで
\begin{verbatim}
    www-ptex@ascii.co.jp
\end{verbatim}
までお願いします。

\end{document}
