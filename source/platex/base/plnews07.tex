%% <2001/10/04>
\documentclass{plnews}

\publicationyear{2001}% 発行年
\publicationmonth{09}% 発行月
\publicationissue{7}% 番号
\author{中野 賢(\texttt{<ken-na at ascii.co.jp>})
     \& 富樫 秀昭(\texttt{<hideak-t at ascii.co.jp>})
}

\begin{document}

\maketitle

\section{この文書について}
この文書は、p\LaTeXe{}\texttt{<2001/09/04>}版について、
前回の版(\texttt{<2000/11/03>})からの更新箇所をまとめたものです。
それ以前の変更点については、\file{plnews*.tex}や\file{Changes.txt}を
参照してください。
\LaTeX{}レベルでの更新箇所は、\LaTeX{}に付属の\file{ltnews*.tex}などを
参照してください。


\section{使用および配付条件の変更}
p\LaTeXe{}の配付および利用条件を「(変更済み)BSDライセンス」にしました。
内容については、\file{COPYRIGHT}ファイルを参照してください。


\section{nidanfloatパッケージ}
nidanfloatパッケージは、最終ページの左右カラムの高さを均一に
して出力するようになっていました。

この機能は、左カラム用に保持している内容と、右カラム用の内容を
一度まとめ、再び2分割するだけの簡略した実装で実現されています。

そのため、左カラムだけで収まる量しかない場合、ページ下部への出力を
指定した(2段抜きでない)フロートは右カラムの下に置かれます。
また、|\newpage|コマンドでカラムを変更しても、
ページ出力時に左カラム用の内容とまとめられ、分割位置が調整されるので、
指定した|\newpage|コマンドの位置でカラムが変わりません。

そこで、最終ページの高さ調整機能を使うかどうかを制御するための
パッケージオプションを導入しました。

自動調整するには、パッケージをロードするときに``balance''オプションを
指定してください。
\begin{verbatim}
\usepackage[balance]{nidanfloat}
\end{verbatim}

逆に、調整しないようにするには、オプション``nobalance''を指定します。
\begin{verbatim}
\usepackage[nobalance]{nidanfloat}
\end{verbatim}

デフォルトは\textbf{nobalance}にしてあります。


\section{その他の主な修正箇所}
次のような不具合の修正や仕様の変更をしました。

\begin{itemize}
\item |\enlargethispage|コマンドを用いた場合、脚注と本文が重なってしまう。
\item |\chpater|コマンドと|\chapter*|コマンドで見出しの出力位置が異なる。
\item |\adjustbaseline|で調整量が合っていない。
\item |\pbox|コマンドでzオプションを指定するとエラーになる。
\item 目次のページ番号の書体を|\rmfamily|から|\normalfont|に変更しました。
\end{itemize}


\section{フォーマットファイル作成時の注意}
現在のp\TeX{}では、8ビットコードの連続を16ビットコードと認識して
しまう場合があります。そのため、フランス語やキリル文字などの
8ビットコードが連続するハイフンパターンはまず使えせん。
例えばcmcyraltパッケージでは、途中でつぎのようなエラーになります。

\begin{verbatim}
(/usr/local/share/texmf/tex/latex/contrib/
other/cmcyralt/rhyphen.tex Russian hyphena
tion
! Bad \patterns.
l.107  . え
           2
?
\end{verbatim}

このときは、``|?|''のプロンプトに対して``|x|''で終了してください。
残念ながら、このハイフンパターンをp\TeX{}で利用することはできません。

そこで、hyphen.cfgを用意して、不用意に他のハイフンパターンを
読み込まないようにしてあります。詳しくはREADME2.txtをご覧ください。

\section{その他}
p\TeX{}やp\LaTeXe{}に関する最新情報は、p\TeX{}ホームページ
\begin{verbatim}
    http://www.ascii.co.jp/pb/ptex
\end{verbatim}
より、入手することができます。

バグ報告やお問い合わせなどは、電子メールで
\begin{verbatim}
    www-ptex@ascii.co.jp
\end{verbatim}
までお願いします。

\end{document}
